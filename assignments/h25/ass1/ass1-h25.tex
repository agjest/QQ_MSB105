% Options for packages loaded elsewhere
\PassOptionsToPackage{unicode}{hyperref}
\PassOptionsToPackage{hyphens}{url}
\PassOptionsToPackage{dvipsnames,svgnames,x11names}{xcolor}
%
\documentclass[
  a4paper,
  DIV=11,
  numbers=noendperiod]{scrartcl}

\usepackage{amsmath,amssymb}
\usepackage{iftex}
\ifPDFTeX
  \usepackage[T1]{fontenc}
  \usepackage[utf8]{inputenc}
  \usepackage{textcomp} % provide euro and other symbols
\else % if luatex or xetex
  \usepackage{unicode-math}
  \defaultfontfeatures{Scale=MatchLowercase}
  \defaultfontfeatures[\rmfamily]{Ligatures=TeX,Scale=1}
\fi
\usepackage{lmodern}
\ifPDFTeX\else  
    % xetex/luatex font selection
\fi
% Use upquote if available, for straight quotes in verbatim environments
\IfFileExists{upquote.sty}{\usepackage{upquote}}{}
\IfFileExists{microtype.sty}{% use microtype if available
  \usepackage[]{microtype}
  \UseMicrotypeSet[protrusion]{basicmath} % disable protrusion for tt fonts
}{}
\makeatletter
\@ifundefined{KOMAClassName}{% if non-KOMA class
  \IfFileExists{parskip.sty}{%
    \usepackage{parskip}
  }{% else
    \setlength{\parindent}{0pt}
    \setlength{\parskip}{6pt plus 2pt minus 1pt}}
}{% if KOMA class
  \KOMAoptions{parskip=half}}
\makeatother
\usepackage{xcolor}
\setlength{\emergencystretch}{3em} % prevent overfull lines
\setcounter{secnumdepth}{-\maxdimen} % remove section numbering
% Make \paragraph and \subparagraph free-standing
\makeatletter
\ifx\paragraph\undefined\else
  \let\oldparagraph\paragraph
  \renewcommand{\paragraph}{
    \@ifstar
      \xxxParagraphStar
      \xxxParagraphNoStar
  }
  \newcommand{\xxxParagraphStar}[1]{\oldparagraph*{#1}\mbox{}}
  \newcommand{\xxxParagraphNoStar}[1]{\oldparagraph{#1}\mbox{}}
\fi
\ifx\subparagraph\undefined\else
  \let\oldsubparagraph\subparagraph
  \renewcommand{\subparagraph}{
    \@ifstar
      \xxxSubParagraphStar
      \xxxSubParagraphNoStar
  }
  \newcommand{\xxxSubParagraphStar}[1]{\oldsubparagraph*{#1}\mbox{}}
  \newcommand{\xxxSubParagraphNoStar}[1]{\oldsubparagraph{#1}\mbox{}}
\fi
\makeatother


\providecommand{\tightlist}{%
  \setlength{\itemsep}{0pt}\setlength{\parskip}{0pt}}\usepackage{longtable,booktabs,array}
\usepackage{calc} % for calculating minipage widths
% Correct order of tables after \paragraph or \subparagraph
\usepackage{etoolbox}
\makeatletter
\patchcmd\longtable{\par}{\if@noskipsec\mbox{}\fi\par}{}{}
\makeatother
% Allow footnotes in longtable head/foot
\IfFileExists{footnotehyper.sty}{\usepackage{footnotehyper}}{\usepackage{footnote}}
\makesavenoteenv{longtable}
\usepackage{graphicx}
\makeatletter
\newsavebox\pandoc@box
\newcommand*\pandocbounded[1]{% scales image to fit in text height/width
  \sbox\pandoc@box{#1}%
  \Gscale@div\@tempa{\textheight}{\dimexpr\ht\pandoc@box+\dp\pandoc@box\relax}%
  \Gscale@div\@tempb{\linewidth}{\wd\pandoc@box}%
  \ifdim\@tempb\p@<\@tempa\p@\let\@tempa\@tempb\fi% select the smaller of both
  \ifdim\@tempa\p@<\p@\scalebox{\@tempa}{\usebox\pandoc@box}%
  \else\usebox{\pandoc@box}%
  \fi%
}
% Set default figure placement to htbp
\def\fps@figure{htbp}
\makeatother

\KOMAoption{captions}{tableheading}
\makeatletter
\@ifpackageloaded{caption}{}{\usepackage{caption}}
\AtBeginDocument{%
\ifdefined\contentsname
  \renewcommand*\contentsname{Innholdsfortegnelse}
\else
  \newcommand\contentsname{Innholdsfortegnelse}
\fi
\ifdefined\listfigurename
  \renewcommand*\listfigurename{Figuroversikt}
\else
  \newcommand\listfigurename{Figuroversikt}
\fi
\ifdefined\listtablename
  \renewcommand*\listtablename{Tabelloversikt}
\else
  \newcommand\listtablename{Tabelloversikt}
\fi
\ifdefined\figurename
  \renewcommand*\figurename{Figur}
\else
  \newcommand\figurename{Figur}
\fi
\ifdefined\tablename
  \renewcommand*\tablename{Tabell}
\else
  \newcommand\tablename{Tabell}
\fi
}
\@ifpackageloaded{float}{}{\usepackage{float}}
\floatstyle{ruled}
\@ifundefined{c@chapter}{\newfloat{codelisting}{h}{lop}}{\newfloat{codelisting}{h}{lop}[chapter]}
\floatname{codelisting}{Liste}
\newcommand*\listoflistings{\listof{codelisting}{Listeoversikt}}
\makeatother
\makeatletter
\makeatother
\makeatletter
\@ifpackageloaded{caption}{}{\usepackage{caption}}
\@ifpackageloaded{subcaption}{}{\usepackage{subcaption}}
\makeatother

\ifLuaTeX
\usepackage[bidi=basic]{babel}
\else
\usepackage[bidi=default]{babel}
\fi
\babelprovide[main,import]{norsk}
% get rid of language-specific shorthands (see #6817):
\let\LanguageShortHands\languageshorthands
\def\languageshorthands#1{}
\usepackage{bookmark}

\IfFileExists{xurl.sty}{\usepackage{xurl}}{} % add URL line breaks if available
\urlstyle{same} % disable monospaced font for URLs
\hypersetup{
  pdftitle={Assignment 1 H25},
  pdflang={nb},
  colorlinks=true,
  linkcolor={blue},
  filecolor={Maroon},
  citecolor={Blue},
  urlcolor={Blue},
  pdfcreator={LaTeX via pandoc}}


\title{Assignment 1 H25}
\author{}
\date{}

\begin{document}
\maketitle


\subsection{Første oppgave (assignment) høsten
25}\label{fuxf8rste-oppgave-assignment-huxf8sten-25}

Ta utgangspunkt i malen i dokumentet
\href{./msb105_ass_1_25.txt}{msb105\_ass\_1\_25} (arkiver med
filendelsen qmd) og skriv et mini-paper på 3-4 sider med utgangspunkt i
foredraget
\href{https://msb105.netlify.app/introduction/reproducibility/rr-science_h25\#/title-slide}{«Robust
and Reliable Science»}. Dere finner en screencast av dette foredraget
under
\href{https://hvl.instructure.com/courses/32799/external_tools/1650}{Panopto
video} på kursets Canvas-sider.

Oppgaven skal utføres individuelt og mot et Github repo. Opprett først
repo-et på Github før du lager et prosjekt i RStudio. Legg filene
msb105\_ass\_1\_24.qmd, apa7.csl og reproducibility.bib inn i ditt nye
prosjekt.

Krav til oppgaven:

\begin{itemize}
\tightlist
\item
  Det skal utføres minst 15 commits

  \begin{itemize}
  \tightlist
  \item
    Med gode «commit messages» ;-)
  \end{itemize}
\item
  Det skal være brukt minst 20 siteringer vha. siteringssystemet i
  Rstudio

  \begin{itemize}
  \tightlist
  \item
    Fem skal dere har funnet selv, dvs. at de ikke er i
    reproducibility.bib
  \end{itemize}
\item
  Det skal brukes minst to grener som skal merges inn i main

  \begin{itemize}
  \tightlist
  \item
    Husk: skift først til main
  \item
    Så i Terminal:
    \texttt{git\ merge\ \textless{}gren-navn\textgreater{}}
  \end{itemize}
\end{itemize}

Når dere er ferdige kan dere invitere meg (agjest) inn i repoet
(eventuelt før hvis dere støter på problemer jeg kan hjelpe med 8-) )

Når jeg er invitert vil jeg gjøre en fork av repo-et, fikse eventuelle
feil og skrive et kort dokument med kommentarer og tips. Når jeg er
ferdig vil dere få en «merge request» fra Github.

Det er lov å spørre om hjelp! Hvis dere sender epost så bruk MSB105 i
tittelen så er det mindre sjanse for at den blir oversett. Send direkte
til ag@hvl.no og ikke via Canvas. Får dere ikke respons så prøv på ny.
Det er dessverre litt «overload» på eposten min for tiden.

Oppstår det ting som alle kan ha bruk for å vite om vil jeg legge det ut
som Melding her på Canvas.




\end{document}
